\documentclass{beamer}

\title{Modelos Generativos Profundos}
\subtitle{Clase 9: Redes generativas adversarias (parte I)}
\author{Fernando Fêtis Riquelme}
\institute{
    Facultad de Ciencias Físicas y Matemáticas\\
    Universidad de Chile
}
\date{Otoño, 2025}
\titlegraphic{\hfill\includegraphics[height=1.2cm]{fcfm}}

\usetheme{metropolis}
\setbeamercovered{transparent}

\begin{document}

\begin{frame}
    \titlepage
\end{frame}

\begin{frame}{Clase de hoy}
    \tableofcontents
\end{frame}

\section{Revisión tarea ARMs}

\section{Introducción a las GANs}

\begin{frame}{Redes generativas adversarias}
    \begin{itemize}
        \item<1> Formulación de una GAN.
        \item<2> Implementación de una GAN simple.
        \item<3> Implementación de una DCGAN.
    \end{itemize}
\end{frame}

\begin{frame}{Próxima clase}
    En la próxima clase.
    \begin{itemize}
        \item<1> Style transfer usando GANs: Pix2Pix y CycleGAN.
        \item<2> Arquitecturas: SAGAN, U-Net.
        \item<3> Limitaciones de las GANs.
    \end{itemize}
\end{frame}

\begin{frame}
    \centering
    \Large{Modelos Generativos Profundos}\\
    \large{Clase 9: Redes generativas adversarias (parte I)}
\end{frame}

\end{document}