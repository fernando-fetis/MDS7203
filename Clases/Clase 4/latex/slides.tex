\documentclass{beamer}

\title{Modelos Generativos Profundos}
\subtitle{Clase 4: Inferencia en redes bayesianas}
\author{Fernando Fêtis Riquelme}
\institute{
    Facultad de Ciencias Físicas y Matemáticas\\
    Universidad de Chile
}
\date{Otoño, 2025}
\titlegraphic{\hfill\includegraphics[height=1.2cm]{fcfm}}

\usetheme{metropolis}
\setbeamercovered{transparent}

\begin{document}

\begin{frame}
    \titlepage
\end{frame}

\begin{frame}{Clase de hoy}
    \tableofcontents
\end{frame}

\section{Ejemplo inicial}


\begin{frame}{Estimación de parámetros}
    \begin{itemize}
        \item<1> Inferencia, inferencia estadística e inferencia bayesiana.
        \item<2> MAP sobre parámetros desconocidos.
        \item<3> Priors y regularización.
    \end{itemize}
\end{frame}

\begin{frame}{Criterio de máxima verosimilitud}
    \begin{itemize}
        \item<1> Definición.
        \item<2> Relación con divergencia de Kullback-Leibler.
        \item<3> Verosimilitud en modelos de variable latente.
        \item<4> Ejemplo.
    \end{itemize}
\end{frame}


\begin{frame}{Próxima clase}
    En la próxima clase.
    \begin{itemize}
        \item<2> Se introducirán los modelos autorregresivos.
        \item<3> Se implementará un modelo de lenguaje basado en una RNN.
    \end{itemize}
\end{frame}

\begin{frame}
    \centering
    \Large{Modelos Generativos Profundos}\\
    \large{Clase 4: Inferencia en redes bayesianas}
\end{frame}

\end{document}