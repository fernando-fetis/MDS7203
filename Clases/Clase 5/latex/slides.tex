\documentclass{beamer}

\title{Modelos Generativos Profundos}
\subtitle{Clase 5: Introducción a los modelos autorregresivos}
\author{Fernando Fêtis Riquelme}
\institute{
    Facultad de Ciencias Físicas y Matemáticas\\
    Universidad de Chile
}
\date{Otoño, 2025}
\titlegraphic{\hfill\includegraphics[height=1.2cm]{fcfm}}

\usetheme{metropolis}
\setbeamercovered{transparent}

\begin{document}

\begin{frame}
    \titlepage
\end{frame}

\begin{frame}{Clase de hoy}
    \tableofcontents
\end{frame}

\section{Modelos autorregresivos}

\begin{frame}{Modelos autorregresivos}
    \begin{itemize}
        \item<1> Formulación.
        \item<2> Uso en texto.
        \item<3> Redes neuronales recurrentes.
        \item<4> Entrenamiento.
        \item<5> Generación: desde la distribución, greedy sampling, top-$k$ sampling, temperatura, penalización.
    \end{itemize}
\end{frame}

\begin{frame}{Próxima clase}
    En la próxima clase.
    \begin{itemize}
        \item<2> Limitaciones de las arquitecturas recurrentes.
        \item<3> Implementación de un modelo GPT.
    \end{itemize}
\end{frame}

\begin{frame}
    \centering
    \Large{Modelos Generativos Profundos}\\
    \large{Clase 5: Introducción a los modelos autorregresivos}
\end{frame}

\end{document}