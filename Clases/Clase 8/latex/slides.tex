\documentclass{beamer}

\title{Modelos Generativos Profundos}
\subtitle{Clase 8: Arquitectura Transformer y LLMs (parte II)}
\author{Fernando Fêtis Riquelme}
\institute{
    Facultad de Ciencias Físicas y Matemáticas\\
    Universidad de Chile
}
\date{Otoño, 2025}
\titlegraphic{\hfill\includegraphics[height=1.2cm]{fcfm}}

\usetheme{metropolis}
\setbeamercovered{transparent}

\begin{document}

\begin{frame}
    \titlepage
\end{frame}

\begin{frame}{Clase de hoy}
    \tableofcontents
\end{frame}

\section{Pendiente clase anterior}

\begin{frame}{Pendiente clase anterior}
    \begin{itemize}
        \item<1> LoRA.
        \item<2> KV caching.
    \end{itemize}
\end{frame}

\section{Propiedades de los LLMs}

\begin{frame}{Propiedades de los LLMs}
    \begin{itemize}
        \item<1> Revisión de algunos modelos.
        \item<2> Scaling laws.
        \item<3> Propiedades emergentes.
        \item<4> Prompting.
        
    \end{itemize}
\end{frame}

\begin{frame}{Algunos modelos tipo Transformer}
    \begin{itemize}
        \item<1> BERT.
        \item<2> Vision Transformer (ViT).
        \item<3> Chameleon.
        \item<4> CLIP.
    \end{itemize}
\end{frame}

\begin{frame}{Próxima clase}
    En la próxima clase.
    \begin{itemize}
        \item<1> Formulación de una GAN.
        \item<2> Implementación de una GAN.
    \end{itemize}
\end{frame}

\begin{frame}
    \centering
    \Large{Modelos Generativos Profundos}\\
    \large{Clase 8: Arquitectura Transformer y LLMs (parte II)}
\end{frame}

\end{document}