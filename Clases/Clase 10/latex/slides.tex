\documentclass{beamer}

\title{Modelos Generativos Profundos}
\subtitle{Clase10: Redes generativas adversarias (parte II)}
\author{Fernando Fêtis Riquelme}
\institute{
    Facultad de Ciencias Físicas y Matemáticas\\
    Universidad de Chile
}
\date{Otoño, 2025}
\titlegraphic{\hfill\includegraphics[height=1.2cm]{fcfm}}

\usetheme{metropolis}
\setbeamercovered{transparent}

\begin{document}

\begin{frame}
    \titlepage
\end{frame}

\begin{frame}{Clase de hoy}
    \tableofcontents
\end{frame}

\section{Algunas cosas sobre las GANs}

\begin{frame}{Redes generativas adversarias}
    \begin{itemize}
        \item<1> Repaso clase anterior.
        \item<2> Transferencia de estilo: pix2pix, CycleGAN.
        \item<3> Arquitecturas para GAN: DCGAN, SAGAN, U-Net.
        \item<4> Limitaciones de las GANs.
    \end{itemize}
\end{frame}

\begin{frame}{Próxima clase}
    En la próxima clase.
    \begin{itemize}
        \item<1> Formulación de un VAE.
        \item<2> Implementación de un VAE.
    \end{itemize}
\end{frame}

\begin{frame}
    \centering
    \Large{Modelos Generativos Profundos}\\
    \large{Clase 10: Redes generativas adversarias (parte II)}
\end{frame}

\end{document}